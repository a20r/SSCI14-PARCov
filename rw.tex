\documentclass{article}

\usepackage{amssymb,amsmath}

\usepackage{graphicx}

\begin{document}

\section{DBLP:conf/icra/MurphyN11~\cite{DBLP:conf/icra/MurphyN11}}

Uses the ALT algorithm and risk heuristics to guide the search over
probabilistic cost maps. This is the paper that uses random terrain generation
as to make the risk maps.

\section{DBLP:journals/cacm/FournierFC82~\cite{DBLP:journals/cacm/FournierFC82}}

Paper that describes the diamond square algorithm for random risk map
generation.

\section{koenig1994make \cite{koenig1994make}}

Describes how to adapt a reactive planner that can sacrifice the probability of
goal achievement to minimize risk.

\section{beard2006decentralized~\cite{beard2006decentralized}}

A decentralized approach for fixed wing aerial surveillance. Nothing too
special in this paper besides it showing that there is a need for this sort of
thing.

\section{hussein2007effective~\cite{hussein2007effective}}

The paper shows that flocking (so simple local rules) can be used to promote
effective sensor coverage of an area. Also the paper assures the reader that
the robots will avoid contact.

\section{alexis2009coordination~\cite{alexis2009coordination}}

This paper is a good reference and shows why risk sensitive path planning is
needed. Quadrotors are used to perform surveillance using UAVs for forest
fires. The quadrotors use their measurements to avoid random windgusts. This
paper is focused on using sensors to predict the movement of the fire.

\section{bethke2009multi~\cite{bethke2009multi}}

This paper provides a solution to the persistent surveillance problem using a
group of UAVs. The paper extends the problem formulation to account for
stricter communication constraints and stochastic sensor failure models.
Basically this paper shows how a group of UAVs can monitor their health,
account for sensor failures, and still maintain persistent coverage.


\section{michael2011persistent~\cite{michael2011persistent}}

This paper describes the system architecture for persistent sensor coverage
using a swarm of micro aerial vehicles (MAVs)

\section{stump2011multi~\cite{stump2011multi}}

This paper models the persistent surveillance problem as a discrete vehicle
routing problem and uses advances in operations research and logistics to solve
it.

\section{julian2011towards~\cite{julian2011towards}}

This extended abstract discusses a unifying information theoretic framework for
multi-robot exploration and surveillance.

\bibliographystyle{IEEEtran} \bibliography{rw}

\end{document}
