\documentclass{article}

\usepackage{amssymb,amsmath}

\usepackage{graphicx}

\begin{document}

\section{DBLP:conf/icra/MurphyN11~\cite{DBLP:conf/icra/MurphyN11}}

Uses the ALT algorithm and risk heuristics to guide the search over
probabilistic cost maps. This is the paper that uses random terrain generation
as to make the risk maps.

\section{DBLP:journals/cacm/FournierFC82~\cite{DBLP:journals/cacm/FournierFC82}}

Paper that describes the diamond square algorithm for random risk map
generation.

\section{koenig1994make \cite{koenig1994make}}

Describes how to adapt a reactive planner that can sacrifice the probability of
goal achievement to minimize risk.

\section{beard2006decentralized~\cite{beard2006decentralized}}

A decentralized approach for fixed wing aerial surveillance. Nothing too
special in this paper besides it showing that there is a need for this sort of
thing.

\section{hussein2007effective~\cite{hussein2007effective}}

The paper shows that flocking (so simple local rules) can be used to promote
effective sensor coverage of an area. Also the paper assures the reader that
the robots will avoid contact.

\section{alexis2009coordination~\cite{alexis2009coordination}}

This paper is a good reference and shows why risk sensitive path planning is
needed. Quadrotors are used to perform surveillance using UAVs for forest
fires. The quadrotors use their measurements to avoid random windgusts. This
paper is focused on using sensors to predict the movement of the fire.

\section{bethke2009multi~\cite{bethke2009multi}}

This paper provides a solution to the persistent surveillance problem using a
group of UAVs. The paper extends the problem formulation to account for
stricter communication constraints and stochastic sensor failure models.
Basically this paper shows how a group of UAVs can monitor their health,
account for sensor failures, and still maintain persistent coverage.


\section{michael2011persistent~\cite{michael2011persistent}}

This paper describes the system architecture for persistent sensor coverage
using a swarm of micro aerial vehicles (MAVs)

\section{stump2011multi~\cite{stump2011multi}}

This paper models the persistent surveillance problem as a discrete vehicle
routing problem and uses advances in operations research and logistics to solve
it.

\section{julian2011towards~\cite{julian2011towards}}

This extended abstract discusses a unifying information theoretic framework for
multi-robot exploration and surveillance.

\section{Bethke2008~\cite{Bethke2008}}

This paper discusses a methodology for health management for UAVs to perform
persistent surveillance that maximizes performance amidst vehicle failures.

\section{grocholsky2006cooperative~\cite{grocholsky2006cooperative}}

This paper discusses how UAVs and UGVs can be used together to share their best
qualities to track targets. Their results are very scalable and they created a
seamless network of UAVs and UGVs.

\section{kuindersma2013variational~\cite{kuindersma2013variational}}

"The paper presents a Bayesian policy search algorithm for problems with policy
dependent cost variance". The search algorithm allows for runtime adjustments
for risk-sensitivity. The results show that policies can be learned using very
little experimental trials that are flexible and risk sensitive.

\section{Real Experiments}

The quadrotors speed was limited in software to 0.5 m / s in the x-y plane and
0.5 m / s in the z axis. The algorithm ran for 140s and completed 6200
iterations which means the quadrotors had an update frequency of 44Hz. This is
more than acceptable for \emph{in situ} applications.

\section{About the controller}

The PID controller used in simulation is similar to the one used in the AscTec
Pelican. The use of the PID controller in simulation mimics the control stack
on the AscTec Pelican. Firstly, a velocity is determined by computing a
heading, converting it to a unit vector, and multiplying by a desired speed
(the maximum speed). This then gets converted into a waypoint and this waypoint
is fed into the PID controller. This is how the control stack of the AscTec
Pelican works deep down inside.

\section{The computer I used}

The experiments were run on a MacBook Pro with 8GB of RAM and a Intel® Core™
i7-3520M CPU @ 2.90GHz × 4 processor running a 64-bit Ubuntu 14.04.

\bibliographystyle{IEEEtran} \bibliography{rw}

\end{document}
