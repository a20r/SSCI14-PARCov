
\documentclass{article}

\usepackage{amssymb,amsmath}

\begin{document}

\section{Risk Model}

Risk is modelled as a function, $\mathbb{R}^3 \rightarrow [0, 1]$ which is use
to determine the risk at every $(x, y, z)$ location within the search space. In
order to determine risk in 3-space, we obtain a 2-D risk model which models the
risk at the minimum altitude. This function $R_0 : \mathbb{R}^2 \rightarrow [0,
1]$ determines what the risk would be at the minimum altitude. This is assumed
to be given to the algorithm \emph{a priori} or can be determined at any time
during the iteration of the algorithm. This ground risk, $R_0$, is modelled as
a lookup table rather than a combination of basis functions in order to give a
more generic model for risk that can be used in any use case of the algorithm.
To determine the 3-D risk, $R(x, y, z)$, we perform an exponential decay on the
given ground risk value, $R_0(x, y)$. 3D risk is defined as follows:

$$ R(x, y, z) = R_0(x, y) \cdot \exp{\left(-\frac{z^2}{K \cdot R_0(x,
y)^2}\right)}$$

Even though risk in 3D is evaluated and is not simply a lookup table, one can
be used instead. This representation for 3D risk is ideal for modelling fires,
detection by hostile agents, or any stimuli that would decrease monotonically
as the altitude increases.

For the experiments, we have used nine different scenes for the representation
of the ground risk, $R_0$. To generate these scenes, we used the diamond-square
algorithm~\cite{DBLP:journals/cacm/FournierFC82} to generate random terrain
maps that have values from zero to one.  The diamond-square algorithm generates
realistic random risk scenes that represent the 2D ground level risk as
anticipated. Random terrain maps have also been used
in~\cite{DBLP:conf/icra/MurphyN11}.

\bibliographystyle{IEEEtran} \bibliography{mp,plaku,quads,relatedwork}

\end{document}
